\documentclass{article}
\usepackage{fullpage}
\usepackage[czech]{babel}
\usepackage{amsfonts}
\usepackage[a5paper,margin=25pt]{geometry}
\usepackage{fontspec}
\usepackage{sectsty}
\usepackage{xcolor}
\usepackage{pagecolor}
\usepackage{afterpage}
\usepackage[many]{tcolorbox}
\usepackage{setspace}
\usepackage{multicol}
\usepackage{enumitem}
\usepackage[compact]{titlesec}

\setlist[itemize]{noitemsep}

\pagenumbering{gobble}

\setstretch{1}
\titlespacing{\section}{0pt}{\parskip}{-\parskip}

% definuj jaroška barvy
\definecolor{red}{cmyk}{0.15, 1, 0.85, 0}
\definecolor{blue}{cmyk}{0.7, 0.2, 0, 0.25}

% nastav jaroška fonty
\newfontfamily\Kapitan{Kapitan-Medium}
\setmainfont{OpenSans}

% nastav hodnoty pro hezké boxíky
\tcbset {
    sharp corners,
    colback = white,
    before skip = 0.2cm,
    after skip = 0.5cm
}

% definuj boxík
\newtcolorbox{bluebox}{
  colback = blue,
  boxrule = 0pt
}

\newtcolorbox{nadpisbox}{
  colback = red,
  boxrule = 0pt,
  grow to left by = -3.5cm,
  grow to right by = -3.5cm
}

\newtcolorbox{redbox}{
  colback = red,
  boxrule = 0pt,
  grow to left by = 25pt,
  grow to right by = 25pt,
  right = 22pt,
  left = 22pt,
  enlarge bottom by = -4pt
}



\newcommand{\nadpis}[1]{
  \begin{nadpisbox}
    \centering \section*{\textcolor{white}{#1}}
  \end{nadpisbox}
}

\newcommand{\podnadpis}[1]{
  \subsection*{\textcolor{red}{#1}}
}


\begin{document}
\begin{titlepage}
  \newgeometry{margin=0pt}
  \pagecolor{red}
    \begin{center}
      \vspace*{\fill}

      \textcolor{white}{\fontsize{60}{60} \Kapitan Průvodce\\\vspace{0.2em}prváka}

      \vspace*{\fill}
      \textcolor{white}{\fontsize{20}{20} \Kapitan Jaroška}

      \vspace{0.5em}

      \begin{bluebox}
        \centering \fontsize{15}{15} \Kapitan \textcolor{white}{2024/2025}
      \end{bluebox}

      \vspace{3em}

    \end{center}
\end{titlepage}
\pagecolor{white}


% mapa školy-
Sem patří mapa školy.
\newpage

% kabinety
Sem patří seznam kabinetů.
\newpage

\nadpis{JAK TO U NÁS CHODÍ}
\noindent \podnadpis{Dělení tříd}
\begin{itemize}[leftmargin=10pt]
  \item \textcolor{red}{\textbf{X.A}} --  \textbf{matematicky zaměřená třída}. Většina studentů už jde z tzv. nižšího gymnázia na Příční ulici (prima až kvarta), na vyšším se k nim následně připojí pár dalších studentů
  \item \textcolor{red}{\textbf{X.B}} -- \textbf{všeobecná třída}, taky přicházejí z nižšího gymnázia
  \item \textcolor{red}{\textbf{X.C}} -- \textbf{všeobecná třída}, druhý cizí jazyk je volitelný: vítězí většinou němčina, ale
k dispozici je i španělština, francouzština a ruština
  \item \textcolor{red}{\textbf{X.D}} -- \textbf{všeobecná třída}, druhý cizí jazyk může být pouze němčina
\end{itemize}
\podnadpis{IT}
\begin{itemize}[leftmargin=10pt]
  \item na začátku školního roku dostane každý student svoje přihlašovací údaje k~\textbf{elektronické třídnici EduPage}
  \item také existuje trochu jiné ID, díky kterému se budeš moct \textbf{přihlašovat v informatice} k počítačům na školní síti a taky na \textbf{školní wifinu}
\end{itemize}

\begin{redbox}
  \begin{itemize}[leftmargin=10pt]
    \item[\textcolor{white}{\textbullet}] \textcolor{white}{trochu složitější je to pak s připojováním notebooků na školní wifi: jde to, ale těžko. Ideální je zajít za panem profesorem Blahou (jehož kabinet si můžeš najít na plánku školy) a poprosit ho o zpřístupnění. Trochu se ti pohrabe v notebooku a zařídí ti, že se bez problémů budeš moct připojit a psát si zápisky online třeba na OneNote. -- nelíbí se mi tento text}
  \end{itemize}
\end{redbox}
\begin{itemize}[leftmargin=10pt]
  \item  také existuje \textbf{studentský server Penguin} -- slouží primárně pro projekty studentů. Pokud tedy plánuješ třeba uspořádat školní akci, potřebuješ server pro svoji ZMP a nebo jen sháníš úložiště/pískoviště pro svoje vyžití, \textbf{nechceš platit hosting a nevadí ti trochu starší způsob přístupu k souborům}, obrať se na profesory informatiky, kteří tě pak můžou nasměrovat na \textbf{odpovědné správce serveru}; zpravidla to bývá \textbf{někdo ze studentstva}.
\end{itemize}

\podnadpis{Školní maily}
\begin{itemize}[leftmargin=10pt]
  \item každý profesor má založený \textbf{svůj školní mail ve tvaru prijmeni@jaroska.cz}, na kterém ho můžeš kontaktovat. S tímhle mailem se většinou nespleteš, ale pro sichr je vždycky lepší se přímo dotyčného profesora \textbf{zeptat, jestli náhodou nemá mail na jiné doméně}, kterou používá častěji -- díky tomu můžeš dostat rychlejší odpověď.
  \item  a taky ti sem dáme pár důležitých mailů, které by se ti mohly hodit:
  \begin{itemize}[leftmargin=0pt]
    \item  ředitel: \textcolor{red}{\textbf{herman@jaroska.cz}}
    \item statutární zástupce (ten ‘hlavní’): \textcolor{red}{\textbf{boucnik@jaroska.cz}}
    \item zástupkyně (třeba na záležitosti s ISICem): \textcolor{red}{\textbf{sitarova@jaroska.cz}}
    \item  hospodářka (té napiš při problémech se systémem obědů): \textcolor{red}{\textbf{turcanu@jaroska.cz}}
  \end{itemize}
\end{itemize}

\newpage

\podnadpis{VOLITELNÉ PŘEDMĚTY}

\noindent V rámci možností (a našich osnov) jdeme s dobou -- \textbf{pro 3. a 4. ročník si tedy
můžeš} (respektive musíš) \textbf{zvolit volitelné předměty}. Výčet níže ber s rezervou
(může se cokoli změnit, některé předměty mohou přibýt nebo ubýt). V případě
nízkého zájmu se rovněž \textbf{některé předměty nemusejí otevřít}, ale v posledních
letech se snažíme otevřít skutečně vše, co je v našich silách.

\begin{multicols}{2}
\noindent \textcolor{red}{1. a 2. volitelný předmět (matematická třída bere jeden, všeobecné berou dva; vybírá se na konci druháku na dva roky)}
  \begin{itemize}
    \item 3. cizí jazyk (N, Fr, Š, Ru)
    \item Cambridge Exam Preparation
    \item Cvičení z biologie a chemie
    \item Cvičení z matematiky a fyziky
    \item Dějiny umění
    \item Deskriptivní geometrie
    \item Ekonomika
    \item Informatika a programování
    \item Konverzace ve 2. cizím jazyce
    \item Latina
    \item Molekulární biologie
  \end{itemize}

  \noindent \textcolor{red}{3 a 4. volitelný předmět (matematická třída bere jeden, všeobecné berou dva; vybírá se na konci třeťáku pro maturitní ročník)}
  \begin{itemize}
    \item Biologie 2
    \item Business English
    \item Dějepis 2
    \item Finanční gramotnost
    \item Fyzika 2 (4.B, C nebo D)
    \item Chemie 2
    \item Informatika a programování 2
    \item Konverzace ve 2. cizím jazyce
    (pokud nezvolena dříve)
    \item Matematika 2 (4.B, C nebo D)
    \item Moderní dějiny
    \item Zeměpis 2
  \end{itemize}

  \vfill\null

  \columnbreak

  \noindent \textcolor{red}{Maturitní semináře (bereš dva, vybírají se na maturitní ročník)}
  \begin{itemize}
    \item S. z biologie
    \item S. z českého jazyka a literatury
    \item S. z dějepisu
    \item S. z deskriptivní geometrie (při
    dřívějším zvolení běžné Deskriptivy)
    \item S. z ekonomiky (při dřívějším zvolení předmětu Ekonomika)
    \item S. z fyziky
    \item S. z chemie
    \item S. z informatiky
    \item S. z matematiky (pro 4.B, C a D)
    \item S. z matematických aplikací (4.A)
    \item S. ze společenských věd
    \item S. ze zeměpisu
  \end{itemize}

  \noindent \textcolor{red}{Sport (bereš jeden, vybírá se na konci třeťáku pro maturitní ročník)}
  \begin{itemize}
    \item Fotbal
    \item Plavání
    \item Posilování
    \item Squash
    \item Tanec
    \item Volejbal
  \end{itemize}

  \begin{tcolorbox}[colback=red,boxrule=0pt]
    \textcolor{white}{\footnotesize Poznámky:
     některé předměty jsou označeny pro třídy X.A
      nebo ostatní. Osnovy matematické a všeobecné třídy se trochu liší, a proto jim musí být
      uzpůsoben i repertoár volitelných předmětů.
      Výčet předmětů nemusí být kompletní:
      neustále se snažíme vymýšlet nové smysluplné
      předměty -- hlavně do čtvrťáku. A některé ze
      seznamu se vůbec nemusejí otevřít.}
\end{tcolorbox}
\end{multicols}

\newpage

\podnadpis{ŠKOLNÍ KANCELÁŘ}

Na vyřizování formalit tu máme \textbf{školní kancelář}. \textbf{Najdeš ji ve druhém patře}, jako
navigaci můžeš použít schéma školy na začátku Průvodce Prváka.
Využiješ ji hlavně na:
\begin{itemize}[leftmargin=10pt]
  \item \textbf{potvrzení o studiu} -- na šalinkartu nebo úlevy na daních
  \item \textbf{čip do jídelny} -- o tom ti víc řekneme v odstavci Jídelna (str. 9)
  \item \textbf{čip k výtahu} -- pokud si přivodíš úraz v naší (krásné!) tělocvičně
  \item \textbf{ztráty a nálezy} -- buď tady, nebo na vrátnici
  \item \textbf{hlášení úrazů} -- abychom ti mohli třeba vyjednat odškodné
\end{itemize}

\podnadpis{JEŽDĚNÍ VÝTAHEM}
Schody tu jsou pro každého, výtah ne úplně. Nějaký dobrák profesor tě občas sveze, i když jsi mohoucí, ale je jich menšina (jak by to tady pak vypadalo), a \textbf{proto se povolení na použití výtahu rozdávají až na základě nějakého relevantního zranění}. \\
Podání žádosti probíhá tak, že \textbf{poprosíš svého třídního učitele o podepsaný formulář}, s tím budeš muset \textbf{skočit do kanceláře} (což se ti se zlomenou nohou možná bude dělat blbě), kde ti sdělí \textbf{tajný kód}, \textbf{který pak budeš muset říct profesoru Blahovi a na tvůj
obědový čip ti nahraje kód na výtah} -- budeš tak mít už dva důvody ho neztratit. \\
Výtah samotný se pak používá tak, že se \textbf{prvně pípneš čipem na bílou krabičku
vlevo od výtahových dveří a až pak si zmáčkneš tlačítko přivolání}. \\
Ve zkratce: nemohoucí a profesoři jezdí výtahem, ostatní nejezdí (a nebo jezdí jen
velmi opatrně). Přejeme pěknou jízdu!

\begin{redbox}
  \textcolor{white}{\footnotesize Poznámka: Ne že bys nejen nesměl jezdit zdravý výtahem, ono to taky dost dobře nejde. Bez oprávnění na čipu tě výtah prostě nebude poslouchat a nepřijede ti. Když už je mysteriózně otevřen na patře prázdný, nikdo ti asi nebude fyzicky bránit do něj naskočit a odjet – ale pokud tě při tom načapá profesor, který to nemá rád, dostaneš bídu. Takže bacha.}
\end{redbox}

\podnadpis{a co ISIC?}
Být studentem je dřina, když ti to průvodčí nevěří. Za drobnou úplatu ti tu však samozřejmě \textbf{vystavíme studentský průkaz ISIC}. Stačí vyplnit žádost (kterou najdeš na {\bf jaroska.cz} v příslušné sekci a nebo v kanceláři na stole), \textbf{předat ji paní zástupkyni Sítařové} a na její mail (\textbf{sitarova@jaroska.cz}) pošli reprezentativní \textbf{fotku} o rozlišení minimálně 300 x 360 px. Nechceme fotky naskenované z dokladů (ačkoli normální headshots jsou v pohodě), divně oříznuté, rozmazané nebo jinak
nekvalitní.

\end{document}
