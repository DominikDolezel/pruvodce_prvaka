\documentclass{article}
\usepackage{fullpage}
\usepackage[czech]{babel}
\usepackage{amsfonts}
\usepackage[a5paper,margin=25pt]{geometry}
\usepackage{fontspec}
\usepackage{sectsty}
\usepackage{xcolor}
\usepackage{pagecolor}
\usepackage{afterpage}
\usepackage[many]{tcolorbox}
\usepackage{setspace}
\usepackage{multicol}
\usepackage{enumitem}
\usepackage[compact]{titlesec}

\setlist[itemize]{noitemsep}

\pagenumbering{gobble}

\setstretch{1}

% definuj jaroška barvy
\definecolor{red}{cmyk}{0.15, 1, 0.85, 0}
\definecolor{blue}{cmyk}{0.7, 0.2, 0, 0.25}

% nastav jaroška fonty
\newfontfamily\Kapitan{Kapitan-Medium}
\allsectionsfont{\Kapitan}
\setmainfont{OpenSans}

% nastav hodnoty pro hezké boxíky
\tcbset {
    sharp corners,
    colback = white,
    before skip = 0.2cm,
    after skip = 0.5cm
}

% definuj boxík
\newtcolorbox{bluebox}{
  colback = blue,
  boxrule = 0pt
}

\newtcolorbox{nadpisbox}{
  colback = red,
  boxrule = 0pt,
  grow to left by = -3.5cm,
  grow to right by = -3.5cm
}

\newtcolorbox{redbox}{
  colback = red,
  boxrule = 0pt,
  grow to left by = 25pt,
  grow to right by = 25pt,
  right = 22pt,
  left = 22pt,
  enlarge bottom by = -4pt
}

\newcommand{\nadpis}[1]{
  \begin{nadpisbox}
    \centering \section*{\textcolor{white}{#1}}
  \end{nadpisbox}
}

\newcommand{\podnadpis}[1]{
  \subsection*{\textcolor{red}{#1}}
}


\begin{document}
\begin{titlepage}
  \newgeometry{margin=0pt}
  \pagecolor{red}
    \begin{center}
      \vspace*{\fill}

      \textcolor{white}{\fontsize{60}{60} \Kapitan Průvodce\\\vspace{0.2em}prváka}

      \vspace*{\fill}
      \textcolor{white}{\fontsize{20}{20} \Kapitan Jaroška}

      \vspace{0.5em}

      \begin{bluebox}
        \centering \fontsize{15}{15} \Kapitan \textcolor{white}{2024/2025}
      \end{bluebox}

      \vspace{3em}

    \end{center}
\end{titlepage}
\pagecolor{white}


% mapa školy-
Sem patří mapa školy.
\newpage

% kabinety
Sem patří seznam kabinetů.
\newpage

\nadpis{Jak to u nás chodí?}
\noindent \podnadpis{Dělení tříd}
\begin{itemize}[leftmargin=10pt]
  \item \textcolor{red}{\textbf{X.A}} --  \textbf{matematicky zaměřená třída}. Většina studentů už jde z tzv. nižšího gymnázia na Příční ulici (prima až kvarta), na vyšším se k nim následně připojí pár dalších studentů
  \item \textcolor{red}{\textbf{X.B}} -- \textbf{všeobecná třída}, taky přicházejí z nižšího gymnázia
  \item \textcolor{red}{\textbf{X.C}} -- \textbf{všeobecná třída}, druhý cizí jazyk je volitelný: vítězí většinou němčina, ale
k dispozici je i španělština, francouzština a ruština
  \item \textcolor{red}{\textbf{X.D}} -- \textbf{všeobecná třída}, druhý cizí jazyk může být pouze němčina
\end{itemize}
\podnadpis{IT}
\begin{itemize}[leftmargin=10pt]
  \item na začátku školního roku dostane každý student svoje přihlašovací údaje k~\textbf{elektronické třídnici EduPage}
  \item také existuje trochu jiné ID, díky kterému se budeš moct \textbf{přihlašovat v informatice} k počítačům na školní síti a taky na \textbf{školní wifinu}
\end{itemize}

\begin{redbox}
  \begin{itemize}[leftmargin=10pt]
    \item[\textcolor{white}{\textbullet}] \textcolor{white}{trochu složitější je to pak s připojováním notebooků na školní wifi: jde to, ale těžko. Ideální je zajít za panem profesorem Blahou (jehož kabinet si můžeš najít na plánku školy) a poprosit ho o zpřístupnění. Trochu se ti pohrabe v notebooku a zařídí ti, že se bez problémů budeš moct připojit a psát si zápisky online třeba na OneNote. -- nelíbí se mi tento text}
  \end{itemize}
\end{redbox}
\begin{itemize}[leftmargin=10pt]
  \item  také existuje \textbf{studentský server Penguin} -- slouží primárně pro projekty studentů. Pokud tedy plánuješ třeba uspořádat školní akci, potřebuješ server pro svoji ZMP a nebo jen sháníš úložiště/pískoviště pro svoje vyžití, \textbf{nechceš platit hosting a nevadí ti trochu starší způsob přístupu k souborům}, obrať se na profesory informatiky, kteří tě pak můžou nasměrovat na \textbf{odpovědné správce serveru}; zpravidla to bývá \textbf{někdo ze studentstva}.
\end{itemize}

\podnadpis{Školní maily}
\begin{itemize}[leftmargin=10pt]
  \item každý profesor má založený \textbf{svůj školní mail ve tvaru prijmeni@jaroska.cz}, na kterém ho můžeš kontaktovat. S tímhle mailem se většinou nespleteš, ale pro sichr je vždycky lepší se přímo dotyčného profesora \textbf{zeptat, jestli náhodou nemá mail na jiné doméně}, kterou používá častěji -- díky tomu můžeš dostat rychlejší odpověď.
  \item  a taky ti sem dáme pár důležitých mailů, které by se ti mohly hodit:
  \begin{itemize}[leftmargin=0pt]
    \item  ředitel: \textcolor{red}{\textbf{herman@jaroska.cz}}
    \item statutární zástupce (ten ‘hlavní’): \textcolor{red}{\textbf{boucnik@jaroska.cz}}
    \item zástupkyně (třeba na záležitosti s ISICem): \textcolor{red}{\textbf{sitarova@jaroska.cz}}
    \item  hospodářka (té napiš při problémech se systémem obědů): \textcolor{red}{\textbf{turcanu@jaroska.cz}}
  \end{itemize}
\end{itemize}

\end{document}
