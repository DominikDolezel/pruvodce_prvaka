\documentclass{article}
\usepackage{fullpage}
\usepackage[czech]{babel}
\usepackage{amsfonts}
\usepackage[a5paper,margin=25pt]{geometry}
\usepackage{fontspec}
\usepackage{sectsty}
\usepackage{xcolor}
\usepackage{pagecolor}
\usepackage{afterpage}
\usepackage[many]{tcolorbox}
\usepackage{setspace}
\usepackage{multicol}
\usepackage{enumitem}
\usepackage[compact]{titlesec}
\usepackage{graphicx}
\usepackage{caption}
\usepackage{mdframed}

\setlist[itemize]{noitemsep}

\pagenumbering{gobble}

\setstretch{1}
\titlespacing{\section}{0pt}{\parskip}{-\parskip}

% definuj jaroška barvy
\definecolor{red}{cmyk}{0.15, 1, 0.85, 0}
\definecolor{blue}{cmyk}{0.7, 0.2, 0, 0.25}
\definecolor{green}{cmyk}{0.7, 0, 0.5, 0.1}

% nastav jaroška fonty
\newfontfamily\Kapitan{Kapitan-Medium}
\setmainfont{OpenSans}

% nastav hodnoty pro hezké boxíky
\tcbset {
    sharp corners,
    colback = white,
    before skip = 0.2cm,
    after skip = 0.5cm
}

% definuj boxík
\newtcolorbox{bluebox}{
  colback = blue,
  boxrule = 0pt,
  grow to left by = 25pt,
  grow to right by = 25pt,
  right = 22pt,
  left = 22pt,
  enlarge bottom by = -4pt
}

\newtcolorbox{greenbox}{
  colback = green,
  boxrule = 0pt,
  grow to left by = 25pt,
  grow to right by = 25pt,
  right = 22pt,
  left = 22pt,
  enlarge bottom by = -4pt
}

\newtcolorbox{nadpisbox}{
  colback = red,
  boxrule = 0pt,
  grow to left by = -3.5cm,
  grow to right by = -3.5cm
}

\newtcolorbox{nadpisboxf}{
  colback = blue,
  boxrule = 0pt,
  grow to left by = -3cm,
  grow to right by = -3cm
}

\newtcolorbox{nadpisboxs}{
  colback = green,
  boxrule = 0pt,
  grow to left by = -4.93cm,
  grow to right by = -4.93cm
}

\newtcolorbox{redbox}{
  colback = red,
  boxrule = 0pt,
  grow to left by = 25pt,
  grow to right by = 25pt,
  right = 22pt,
  left = 22pt,
  enlarge bottom by = -4pt
}

\newcommand{\nadpis}[1]{
  \begin{nadpisbox}
    \centering \section*{\textcolor{white}{#1}}
  \end{nadpisbox}
}

\newcommand{\nadpisf}[1]{
  \begin{nadpisboxf}
    \centering \section*{\textcolor{white}{#1}}
  \end{nadpisboxf}
}

\newcommand{\nadpiss}[1]{
  \begin{nadpisboxs}
    \centering \section*{\textcolor{white}{#1}}
  \end{nadpisboxs}
}

\newcommand{\podnadpis}[1]{
  \subsection*{\textcolor{red}{#1}}
}

\newcommand{\podnadpisf}[1]{
  \subsection*{\textcolor{blue}{#1}}
}

\newcommand{\podnadpiss}[1]{
  \subsection*{\textcolor{green}{#1}}
}

\begin{document}
\begin{titlepage}
  \newgeometry{margin=0pt}
  \pagecolor{red}
    \begin{center}
      \vspace*{\fill}

      \textcolor{white}{\fontsize{60}{60} \Kapitan Průvodce\\\vspace{0.2em}prváka}

      \vspace*{\fill}
      \textcolor{white}{\fontsize{20}{20} \Kapitan Jaroška}

      \vspace{0.5em}

      \begin{bluebox}
        \centering \fontsize{15}{15} \Kapitan \textcolor{white}{2024/2025}
      \end{bluebox}

      \vspace{3em}

    \end{center}
\end{titlepage}
\pagecolor{white}


% mapa školy-
Sem patří mapa školy.
\newpage

% kabinety
Sem patří seznam kabinetů.
\newpage

\nadpis{JAK TO U NÁS CHODÍ}
\noindent \podnadpis{Dělení tříd}
\begin{itemize}[leftmargin=10pt]
  \item \textcolor{red}{\textbf{X.A}} --  \textbf{matematicky zaměřená třída}. Většina studentů už jde z tzv. nižšího gymnázia na Příční ulici (prima až kvarta), na vyšším se k nim následně připojí pár dalších studentů
  \item \textcolor{red}{\textbf{X.B}} -- \textbf{všeobecná třída}, taky přicházejí z nižšího gymnázia
  \item \textcolor{red}{\textbf{X.C}} -- \textbf{všeobecná třída}, druhý cizí jazyk je volitelný: vítězí většinou němčina, ale
k dispozici je i španělština, francouzština a ruština
  \item \textcolor{red}{\textbf{X.D}} -- \textbf{všeobecná třída}, druhý cizí jazyk může být pouze němčina
\end{itemize}
\podnadpis{IT}
\begin{itemize}[leftmargin=10pt]
  \item na začátku školního roku dostane každý student svoje přihlašovací údaje k~\textbf{elektronické třídnici EduPage}
  \item také existuje trochu jiné ID, díky kterému se budeš moct \textbf{přihlašovat v informatice} k počítačům na školní síti a taky na \textbf{školní wifinu}
\end{itemize}

\begin{redbox}
  \begin{itemize}[leftmargin=10pt]
    \item[\textcolor{white}{\textbullet}] \textcolor{white}{trochu složitější je to pak s připojováním notebooků na školní wifi: jde to, ale těžko. Ideální je zajít za panem profesorem Blahou (jehož kabinet si můžeš najít na plánku školy) a poprosit ho o zpřístupnění. Trochu se ti pohrabe v notebooku a zařídí ti, že se bez problémů budeš moct připojit a psát si zápisky online třeba na OneNote. -- nelíbí se mi tento text}
  \end{itemize}
\end{redbox}
\begin{itemize}[leftmargin=10pt]
  \item  také existuje \textbf{studentský server Penguin} -- slouží primárně pro projekty studentů. Pokud tedy plánuješ třeba uspořádat školní akci, potřebuješ server pro svoji ZMP a nebo jen sháníš úložiště/pískoviště pro svoje vyžití, \textbf{nechceš platit hosting a nevadí ti trochu starší způsob přístupu k souborům}, obrať se na profesory informatiky, kteří tě pak můžou nasměrovat na \textbf{odpovědné správce serveru}; zpravidla to bývá \textbf{někdo ze studentstva}.
\end{itemize}

\podnadpis{Školní maily}
\begin{itemize}[leftmargin=10pt]
  \item každý profesor má založený \textbf{svůj školní mail ve tvaru prijmeni@jaroska.cz}, na kterém ho můžeš kontaktovat. S tímhle mailem se většinou nespleteš, ale pro sichr je vždycky lepší se přímo dotyčného profesora \textbf{zeptat, jestli náhodou nemá mail na jiné doméně}, kterou používá častěji -- díky tomu můžeš dostat rychlejší odpověď.
  \item  a taky ti sem dáme pár důležitých mailů, které by se ti mohly hodit:
  \begin{itemize}[leftmargin=0pt]
    \item  ředitel: \textcolor{red}{\textbf{herman@jaroska.cz}}
    \item statutární zástupce (ten ‘hlavní’): \textcolor{red}{\textbf{boucnik@jaroska.cz}}
    \item zástupkyně (třeba na záležitosti s ISICem): \textcolor{red}{\textbf{sitarova@jaroska.cz}}
    \item  hospodářka (té napiš při problémech se systémem obědů): \textcolor{red}{\textbf{turcanu@jaroska.cz}}
  \end{itemize}
\end{itemize}

\newpage

\podnadpis{VOLITELNÉ PŘEDMĚTY}

\noindent V rámci možností (a našich osnov) jdeme s dobou -- \textbf{pro 3. a 4. ročník si tedy
můžeš} (respektive musíš) \textbf{zvolit volitelné předměty}. Výčet níže ber s rezervou
(může se cokoli změnit, některé předměty mohou přibýt nebo ubýt). V případě
nízkého zájmu se rovněž \textbf{některé předměty nemusejí otevřít}, ale v posledních
letech se snažíme otevřít skutečně vše, co je v našich silách.

\begin{multicols}{2}
\noindent \textcolor{red}{1. a 2. volitelný předmět (matematická třída bere jeden, všeobecné berou dva; vybírá se na konci druháku na dva roky)}
  \begin{itemize}
    \item 3. cizí jazyk (N, Fr, Š, Ru)
    \item Cambridge Exam Preparation
    \item Cvičení z biologie a chemie
    \item Cvičení z matematiky a fyziky
    \item Dějiny umění
    \item Deskriptivní geometrie
    \item Ekonomika
    \item Informatika a programování
    \item Konverzace ve 2. cizím jazyce
    \item Latina
    \item Molekulární biologie
  \end{itemize}

  \noindent \textcolor{red}{3 a 4. volitelný předmět (matematická třída bere jeden, všeobecné berou dva; vybírá se na konci třeťáku pro maturitní ročník)}
  \begin{itemize}
    \item Biologie 2
    \item Business English
    \item Dějepis 2
    \item Finanční gramotnost
    \item Fyzika 2 (4.B, C nebo D)
    \item Chemie 2
    \item Informatika a programování 2
    \item Konverzace ve 2. cizím jazyce
    (pokud nezvolena dříve)
    \item Matematika 2 (4.B, C nebo D)
    \item Moderní dějiny
    \item Zeměpis 2
  \end{itemize}

  \vfill\null

  \columnbreak

  \noindent \textcolor{red}{Maturitní semináře (bereš dva, vybírají se na maturitní ročník)}
  \begin{itemize}
    \item S. z biologie
    \item S. z českého jazyka a literatury
    \item S. z dějepisu
    \item S. z deskriptivní geometrie (při
    dřívějším zvolení běžné Deskriptivy)
    \item S. z ekonomiky (při dřívějším zvolení předmětu Ekonomika)
    \item S. z fyziky
    \item S. z chemie
    \item S. z informatiky
    \item S. z matematiky (pro 4.B, C a D)
    \item S. z matematických aplikací (4.A)
    \item S. ze společenských věd
    \item S. ze zeměpisu
  \end{itemize}

  \noindent \textcolor{red}{Sport (bereš jeden, vybírá se na konci třeťáku pro maturitní ročník)}
  \begin{itemize}
    \item Fotbal
    \item Plavání
    \item Posilování
    \item Squash
    \item Tanec
    \item Volejbal
  \end{itemize}

  \begin{tcolorbox}[colback=red,boxrule=0pt]
    \textcolor{white}{\footnotesize Poznámky:
     některé předměty jsou označeny pro třídy X.A
      nebo ostatní. Osnovy matematické a všeobecné třídy se trochu liší, a proto jim musí být
      uzpůsoben i repertoár volitelných předmětů.
      Výčet předmětů nemusí být kompletní:
      neustále se snažíme vymýšlet nové smysluplné
      předměty -- hlavně do čtvrťáku. A některé ze
      seznamu se vůbec nemusejí otevřít.}
\end{tcolorbox}
\end{multicols}

\newpage

\podnadpis{ŠKOLNÍ KANCELÁŘ}

Na vyřizování formalit tu máme \textbf{školní kancelář}. \textbf{Najdeš ji ve druhém patře}, jako
navigaci můžeš použít schéma školy na začátku Průvodce Prváka.
Využiješ ji hlavně na:
\begin{itemize}[leftmargin=10pt]
  \item \textbf{potvrzení o studiu} -- na šalinkartu nebo úlevy na daních
  \item \textbf{čip do jídelny} -- o tom ti víc řekneme v odstavci Jídelna (str. 9)
  \item \textbf{čip k výtahu} -- pokud si přivodíš úraz v naší (krásné!) tělocvičně
  \item \textbf{ztráty a nálezy} -- buď tady, nebo na vrátnici
  \item \textbf{hlášení úrazů} -- abychom ti mohli třeba vyjednat odškodné
\end{itemize}

\podnadpis{JEŽDĚNÍ VÝTAHEM}
Schody tu jsou pro každého, výtah ne úplně. Nějaký dobrák profesor tě občas sveze, i když jsi mohoucí, ale je jich menšina (jak by to tady pak vypadalo), a \textbf{proto se povolení na použití výtahu rozdávají až na základě nějakého relevantního zranění}. \\
Podání žádosti probíhá tak, že \textbf{poprosíš svého třídního učitele o podepsaný formulář}, s tím budeš muset \textbf{skočit do kanceláře} (což se ti se zlomenou nohou možná bude dělat blbě), kde ti sdělí \textbf{tajný kód}, \textbf{který pak budeš muset říct profesoru Blahovi a na tvůj
obědový čip ti nahraje kód na výtah} -- budeš tak mít už dva důvody ho neztratit. \\
Výtah samotný se pak používá tak, že se \textbf{prvně pípneš čipem na bílou krabičku
vlevo od výtahových dveří a až pak si zmáčkneš tlačítko přivolání}. \\
Ve zkratce: nemohoucí a profesoři jezdí výtahem, ostatní nejezdí (a nebo jezdí jen
velmi opatrně). Přejeme pěknou jízdu!

\begin{redbox}
  \textcolor{white}{\footnotesize Poznámka: Ne že bys nejen nesměl jezdit zdravý výtahem, ono to taky dost dobře nejde. Bez oprávnění na čipu tě výtah prostě nebude poslouchat a nepřijede ti. Když už je mysteriózně otevřen na patře prázdný, nikdo ti asi nebude fyzicky bránit do něj naskočit a odjet – ale pokud tě při tom načapá profesor, který to nemá rád, dostaneš bídu. Takže bacha.}
\end{redbox}

\podnadpis{a co ISIC?}
Být studentem je dřina, když ti to průvodčí nevěří. Za drobnou úplatu ti tu však samozřejmě \textbf{vystavíme studentský průkaz ISIC}. Stačí vyplnit žádost (kterou najdeš na {\bf jaroska.cz} v příslušné sekci a nebo v kanceláři na stole), \textbf{předat ji paní zástupkyni Sítařové} a na její mail (\textbf{sitarova@jaroska.cz}) pošli reprezentativní \textbf{fotku} o rozlišení minimálně 300 x 360 px. Nechceme fotky naskenované z dokladů (ačkoli normální headshots jsou v pohodě), divně oříznuté, rozmazané nebo jinak
nekvalitní.

\newpage

\podnadpis{ŠKOLNÍ VÝLETY}
Ačkoli nám do toho Covid v posledních letech házel vidle, jezdíme i na výlety.
\textbf{V každém ročníku tě čeká jeden} -- \textbf{tady je jejich přehled}.

\begin{itemize}[leftmargin=10pt]
  \item \textcolor{red}{\textbf{Prvák a druhák:}} \textbf{třídenní výlet}, který si volí třída, lze protáhnout i do víkendu
  \item \textcolor{red}{\textbf{Třeťák:}} každý rok jsou připraveny \textbf{čtyři sportovní kurzy}, ze kterých si každý student zvolí jeden: \textbf{vodácký kurz}, \textbf{vysokohorská turistika}, \textbf{zájezd do Chorvatska}, \textbf{windsurfing}
  \item \textcolor{red}{\textbf{Čtvrťák:}} \textbf{třídenní literárně-historická exkurze do Prahy}
\end{itemize}

\begin{redbox}


  \begin{minipage}{0.32\linewidth}
    \includegraphics[width=\linewidth]{dulezite.jpg}
    \centering \textcolor{white}{aa}
  \end{minipage}
  \hfill
  \begin{minipage}{0.32\linewidth}
    \includegraphics[width=\linewidth]{dulezite.jpg}
    \centering \textcolor{white}{aa}
  \end{minipage}
  \hfill
  \begin{minipage}{0.32\linewidth}
    \includegraphics[width=\linewidth]{dulezite.jpg}
    \centering \textcolor{white}{aa}
  \end{minipage}
\end{redbox}

\podnadpis{MIMOŠKOLNÍ AKTIVITY}
\begin{itemize}[leftmargin=10pt]
  \item \textbf{Česká středoškolská unie (ČSU)} zastupuje a hájí zájmy českých středoškoláků,  podporuje všeobecné vzdělávání, metodickou pestrost a inovace ve výuce, zasazuje se o zlepšování vzdělávací soustavy jako celku
  \item \textbf{Soutěž a podnikej} je organizace nabízející životní příležitost rozvinout svůj nápad v reálný podnikatelský záměr
  \item \textbf{Yoda mentorship} rozvíjí nadané a aktivní středoškoláky prostřednictvím individuálního mentoringu
  \item \textbf{Bakala foundation} podporuje studium talentovaných studentů na prestižních zahraničních univerzitách
\end{itemize}
...a aby toho nebylo málo:

\begin{itemize}[leftmargin=10pt]
    \item \textbf{Hudebna jako zkušebna} -- Ať už si chceš založit fungl novou kapelu nebo si jen zahrát, můžeš se domluvit s profesorkou Ambrozovou, která tobě a tvojí partě ráda půjčí klíče od hudebny, kde je plno nástrojů, a budete tak moct povyučování zkoušet!
\end{itemize}

\newpage

\nadpisf{FORMÁLNÍ ZÁLEŽITOSTI}
\noindent \podnadpisf{JÍDELNA}
V přízemí máme samozřejmě \textbf{jídelnu}. V ní je každý školní den na výběr ze 3 obědů, zpravidla dvou teplých jídel a jednoho salátu v krabičce. Ceny se pohybují v rozmezí \textbf{36 až 44 korun}, v ceně je i polévka a pití.

\begin{itemize}[leftmargin=10pt]
  \item \textbf{objednávat lze dvěma způsoby}: nejlepší je, když půjdeš na \textit{jidelna.jaroska.cz} a objednáš si tam, máme ale i fyzickou ‘narážečku’ před jídelnou, kde se naskenuješ čipem a můžeš objednávat
  \item \textbf{nejzažší termín} pro objednání oběda je \textbf{do 10 hodin} předchozího pracovního dne (takže oběd na pondělí si musíš objednat už v pátek ráno)
\end{itemize}

\podnadpisf{BURZA}
Stejný termín jako přihlášení obědů má i \textbf{odhlášení obědů}. Pokud onemocníš, \textbf{můžeš odhlásit oběd taky nejpozději do 10 hodin} předchozího pracovního dne. Pokud to nestihneš, můžeš ho pořád hodit do burzy. Ta umožňuje, že kdyby si nějaký nešťastník \textbf{nestihl oběd objednat do termínu}, může ho najít v burze od někoho, kdo si ho naopak \textbf{nestihl odhlásit do termínu}.

\podnadpisf{OBĚDVÁNÍ}
Jídelnu máme denně otevřenou \textbf{od 11.15 do 14.00}. V téhle době o vytyčené přestávce (ačkoli pokud půjdeš o volné hodině, nikomu asi vadit nebudeš) si
můžeš vyzvednout svůj oběd. Pro bezkonfliktní předání doporučujeme být \textbf{přezutý} (na to jsou kuchařky fakt háklivé) a batoh a bundu si nechej v nových držácích před jídelnou. \textbf{Vstoupíš dveřmi na konci chodby}, vezmeš si tácek, příbor, můžeš přibrat cestou i pití s polívkou. U výdeje se pípneš čipem a po vydání obědu už je čas zmizet ke stolu.

\podnadpisf{ZAPOMENUTÝ ČIP}
To se taky stává. Nic se neděje, stačí v průběhu dopoledne \textbf{dojít za paní
sekretářkou}. Ta ověří tvoji volbu oběda v systému, vypíše ti \textbf{příslušný lísteček
a tím se prokážeš} v jídelně.

\begin{bluebox}
  \textcolor{white}{\footnotesize
  ‘v jídelně jsme si všichni rovni’ aneb jeden podpultový tip:\\
  v případě úplně nejhoršího scénáře, kdy si zapomeneš objednat jídlo a v burze nic není, můžeš zkusit dojít čtvrt hodiny před druhou a hodit na kuchařky psí oči, třeba ti něco dají. a i když se nevyvede s hlavním chodem, většinou ti dovolí vzít si polívku. (ale od nás to nemáš!)}
\end{bluebox}

\pagebreak
\podnadpisf{ABSENCE}
Nebudeme to okecávat. Tady je flowchart, který je lepší než tisíc slov.

sem patří flowchart

\pagebreak

\podnadpisf{ZNÁMKOVÁNÍ}
Každý učitel má vymyšlené své způsoby známkování. \textbf{Většina se drží u staré dobré škály 1-5}, někteří ale dávají body a jiní zase procenta. Kvůli českému školnímu systému se ale \textbf{ve výsledku vše převede na klasické známky} (a i proto u těchto ‘alternativních’ způsobů budeš na začátku roku poučen, kolik procent je jaká známka).

\begin{bluebox}
\textcolor{white}{\subsection*{EDUPAGE (většinou jen EP)}
\textbf{Naše elektronická třídnice.} Zavedli jsme ji na začátku roku 2020, takže \textbf{prošla doslova zkouškou ohněm} -- a úspěšně (i když to je diskutabilní). Občas na EP nadáváme, ale to snad ani
nejde jinak. \\
Faktem je, že tam najdeš \textbf{úplně všechno} -- rozvrhy, suplování, úkoly, hlasování, známky, absence, prezentace... a to všechno na kvadrát. \\
A taky si tam můžeš psát s učiteli. Jedna z prvních věcí by tedy mělo být milostné psaní Gábině Chovancové. \\
Přidat rant o tom jak lze naimportovat rozvrh do google kalendáře. Actually to má být v modrém boxíku, toto celé ne. Je to designová chyba. Boxík je pro poznámky či doplňující informace.
}
\end{bluebox}

\podnadpisf{INDIVIDUÁLNÍ STUDIJNÍ PLÁN (též INDIVIDUÁL či ISP)}
Když se třeba začneš angažovat v divadle, jiném umění, či sportu, tak to pochopitelně neznamená, že tě vyhodíme. \textbf{Na stránkách školy najdeš tři lejstra, které á ty češtináři musíš pro získání (je to s velkým?) Individuálního studijního plánu vyplnit a předat třídnímu učiteli.} hodilo by se dát sem ten odkaz lol Ten to předá řediteli, který posoudí oprávněnost nároku.
ISP nemusí dávat podmínky pro úplně všechny předměty. Bohatě stačí, když se
s \textbf{jednotlivými kantory domluvíš} na tom, jak se bude klasifikovat. Ustálená praxe pro studenty na ISP je taková, že si \textbf{jednou za čtvrtletí docházejí napsat opakovací testy} z jednotlivých předmětů. ??? fr? cap

\begin{bluebox}
  \textcolor{white}{\subsection*{OMLUVENÍ Z TĚLOCVIKU}
  To by jen tak nešlo. \textbf{Nejdřív ti doktor musí vydat nějaké lejstro či potvrzení o tom, že do toho tělocviku prostě nemůžeš chodit, pak ho předáš nám a... jsi omluven.} proč je tučná celá ta věta? uncool Bude tam ještě trocha papírování, ale snažíme se vycházet vstříc (nebudeme, proboha, nikoho nutit cvičit se sádrou). V prváku také podstupujeme \textbf{test plavecké zdatnosti}, kterému se nevyhneš bez uvolnění. Pokud na něj v prváku nepřijdeš, půjdeš ve druháku. Tak důslední jsme. (Pokud nepřijdeš v druháku, přijdeš ve třeťáku. Pokud nepřijdeš ve třeťáku a zvolíš si vodácký kurz, budeš zkoušky provádět na místě před všemi svými spolužáky.)
  }
\end{bluebox}

\podnadpisf{A CO JINAK?}
Průvodce prváka (mám to dávat tim fancy fontem jak to maj oni? mně to přijde nadbytečný) tady bohužel není od toho, aby poskytnul vyčerpávající výčet takovýchhle situací. \textbf{Pokud nebudeš něco vědět, prostě zajdi za třídním.} opět-- nemám rád tučnění celých vět Vždycky všecko vyřeší. Ne nadarmo jim říkáme matky/(mezera? nevim)otcové třídní.

hej to se tam vleze, bude toho míň slibuju
\pagebreak
\nadpiss{SOUTĚŽE}
\noindent Jestli něco, tak na Jarošce je extrémně pozitivní, že tě \textbf{nechá dělat věci navíc} (a často tě v nich dokonce podporuje) -- tato věta není česky sry. Rozhodli jsme se tedy zkompilovat \textbf{seznam soutěží a jiných aktivit}, které ti zdejší studium umožňuje. -- ověřit shodu v tom které nevjsem si teď jistej

\podnadpiss{STŘEDOŠKOLSKÁ ODBORNÁ ČINNOST (SOČ)}
SOČ, řečená Sočka, představuje možnost zkusit už na střední škole \textbf{dělat opravdovou vědu}. Cílem Sočky je provést vlastní výzkum a získané poznatky \textbf{prezentovat před náročnou komisí}. Výběr okruhů bývá dostatečně široký, najdeš témata jak technicky, tak humanitně zaměřená. kde je rant o závěrečkách???

\podnadpiss{PŘEDMĚTOVÉ OLYMPIÁDY}
Matematika, fyzika, biologie, chemie, zeměpis, dějepis, český jazyk, ekologie. To je základní paleta předmětových olympiád, ve kterých u nás můžeš zkusit štěstí. V olympiádách najdeš typicky k řešení úlohy složitější, než jsou ty běžně řešené v hodinách. Pro úspěch v nich je potřeba mít komplexní znalosti v daném oboru. Přehled těchto olympiád a dalších soutěží je k mání na \textit{chces-soutezit.cz}.

\podnadpiss{T-EXKURZE}
Dvakrát do roka si vybraná vysokoškolská pracoviště pozvou středoškoláky do svých laboratoří, kde si mohou vyzkoušet, jak se dělá reálná věda. Přihlášení je jednoduché: stačí vyplnit pár otázek na dané téma a připojit motivační dopis (a občas ani to ne). Stojí to za zvážení!
\\\\
\noindent Níže pak už najdeš jen výčet dalších soutěží dle ka-
tegorií (který určitě není vyčerpávající). Jejich hlavní cenou je obvykle dobrý pocit nebo možnost účasti na \textbf{dedikovaném soustředění}, kde se můžeš setkat s obdobně divnými lidmi jako ty sám. Jednotlivé soutěže se liší jak obtížností, tak časovou náročností. Každý si tak může najít svoji optimální, do které se mu bude chtít investovat energie.

\begin{greenbox}
  \textcolor{white}{
  sem patří seznam
  }
\end{greenbox}

\end{document}
